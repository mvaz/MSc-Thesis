\chapter{Summary and discussion} % (fold)
\label{cha:discussion}

This thesis explores properties of the ramping parameter model~\cite{Sindel:2009} for graph models.
In particular, the random graph model by~\cite{} and the configuration model were explored.


The random graph model is a simple model of network formation, which generates networks with different statistical properties than real-world networks, but from which analytic solutions can be derived.
For studying the ramping parameter model, this is an interesting model to explore, since it has very relaxed assumptions:
\begin{itemize}
- every edge exists independently from each other
- the properties of the edges are unclear
\end{itemize}

The configuration model generates networks that follow an input degree distribution.
This is an important property, since degree distributions of real-world networks have properties that are independent of the domain of the network.
For example....



state the assumptions of the thesis

state the results

Open points

- what if the graph is directed?

- what if the graph is not power law?



Further work


Generating functions open the door to dealing with directed graphs with other degree distributions







% chapter discussion (end)