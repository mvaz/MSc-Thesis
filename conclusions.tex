\chapter{Summary and discussion} % (fold)
\label{cha:discussion}

This thesis explores properties the construction process of the ramping parameter model~\cite{Sindel:2009vd} using theoretical results from the random graph model.
To the best of our knowledge, this has not been addressed in the literature.


Usually, credit concentration risk measures take only one of the above aspects into consideration~\cite{lutkebohmert2008concentration}.
By contrast, the ramping parameter model is a measure of concentration risk in credit portfolios that takes both (i) the exposure to the obligors and (ii) the  interdependencies between obligors, which renders it especially interesting.

The ramping parameter model works by considering the correlation matrix of obligors as a graph.
For each value of a ramping parameter $\ramp \in [0,1]$, edges are removed from that graph if their weight (associated correlation or interdependency value) does not exceed the value of the parameter.
The community structure of the reduced graph is then taken into consideration.
Namely, the relative exposure-weight of the connected component with largest exposure $CR(\ramp)$ is tracked for each value of the parameter.
In the original work, the curve is weighted and integrated to provide a final concentration value.
An interesting discussion on which weight curve should be used is also part of the original work.
Unfortunately, the choice of weight curve lacks a theoretical backing from random graph theory, motivating this thesis.


The random graph model is a simple model of network formation that generates networks with different statistical properties than real-world networks, but from which analytic solutions can be derived.
For studying the ramping parameter model, this is an interesting model to explore, since it has very relaxed assumptions:
\begin{itemize}
\item every edge exists independently from each other
\item the probability of each edge is constant across the entire graph
\end{itemize}
and, as already mentioned, largely analytically tractable.


This thesis studies value of the ramping parameter model expected by the random graph model $G_{n,p}$, by making the following assumptions:
\begin{enumerate}
	\item the value of $EAD_i$ is equal for all obligors
	\item the correlations between obligors are statistically independent
\end{enumerate}
As show in chapter~\ref{cha:ramping_parameter_under_the_random_graph_model}, under the random graph model, we show that it is possibly to calculate the expected size of the largest connected component, and, thanks to assumption (1), the expected value of the ramping parameter for each value $\ramp$.
In particular, this result takes into account the overall distribution of the correlation values of the original correlation matrix.



\section{Outlook and future work}
\label{sec:future_work}

Time is both the biggest friend and enemy of such a work.
While it ensures the convergence to a possibly useful result, it leaves many questions unanswered. 

For us, the main unanswered question is to which extent the theoretical curve computed in the end of section~\ref{sub:small_component_sizes} can be used for determining the weight curve (see Equation~\vref{eq:ramping_integral}).
Our hypothesis is that it could serve as some sort of null model and therefore lead to a weight curve and measure of concentration risk that takes the insights from the random graph models into account.

Another question arises as to whether other, more expressive models of random graphs, can be used.
One such model is the configuration model, which can express the properties of graphs with more realistic degree distributions, such as the popular power law distribution.
In the literature research leading to this thesis, we came across theoretical results that describe the
effect of edge removal from graphs with power-law and power-law like degree distributions~\cite{dubois2012effect}.
These could be used in order to derive a theoretical distribution of the ramping parameter curve under the assumptions that certain obligors are intrinsically more densely connected than others.
This is definitely more in line with what is observed in real world portfolios.


Lastly, a further possible extension of this work deals would allow for asymmetric interdependencies between obligors, i.e. the graphs under study would be directed graphs.
The treatment of directed graphs is, however, more challenging than the that of undirected graphs.
In particular, the definition of node degree is not unequivocal~\cite{newman2010networks}, like in the case of undirected graphs.

% The ramping parameter its concern with the study of a measure for concentration risk.
% In a risk management Setting, it would also be important to be able identify the obligors that increase the risk portfolio, or even of providing information regarding new credits / obligors that can be added to the portfolio so that the concentration risk effectively diminishes.


% section future_work (end)

% chapter discussion (end)