\chapter{Summary and discussion} % (fold)
\label{cha:discussion}

This thesis explores properties of the ramping parameter model~\cite{Sindel:2009} for graph models.
In particular, the random graph model by~\cite{} and the configuration model were explored.
The study of the theoretical properties of the ramping parameter is, to the best of our knowledge, not in the literature.


The random graph model is a simple model of network formation, which generates networks with different statistical properties than real-world networks, but from which analytic solutions can be derived.
For studying the ramping parameter model, this is an interesting model to explore, since it has very relaxed assumptions:
\begin{itemize}
\item every edge exists independently from each other
\item the properties of the edges are unclear
\end{itemize}

The configuration model generates networks that follow an input degree distribution.
This is an important property, since degree distributions of real-world networks have properties that are independent of the domain of the network.
For example....



state the assumptions of the thesis

state the results


\section{Future work}
\label{sec:future_work}

What is the most adequate model for studying obligor interdependency?
Are the power law graphs the best?
Is there a correlation between the weight and the degree of the weight matrix?


Directed graphs.
There is a treatment of the directed graphs model with the generating functions.
Theoretical properties of connectedness are, however, cannot easily be derived from them.



Connection to the study of real portfolios.
How to use the studied properties as null models for the ramping parameters of real networks?
Is there a connection to other measures of the concentration risk?


The ramping parameter its concern with the study of a measure for concentration risk.
In a risk management Setting, it would also be important to be able identify the obligors that increase the risk portfolio, or even of providing information regarding new credits / obligors that can be added to the portfolio so that the concentration risk effectively diminishes.


% section future_work (end)

% chapter discussion (end)