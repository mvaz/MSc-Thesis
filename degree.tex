The interest of this work lies in the influence that the connectivity of the network structure of a credit portfolio has on its credit risk.
The connectivity is typically quantified at looking at the number / total weight of the connections of each node, or by measuring how many connections connected nodes share(clustering coefficient).
In this work we will look at the distribution of the degree of the nodes of the network and its influence on the so-called ramping parameter.

We represent by $p_k$ the probability that a randomly chosen vertex has degree $k$, and as in [Newman ...] represent the distributions by a generating function.

\begin{equation}
	G_0(x) = \sum_{k=0}^{\infty} p_k x^k,
\end{equation}

Using this, the average degree of the $z$ of a node is given by
\begin{equation}
	z = \lt k \gt = \sum_k k p_k = G'_0(1)
\end{equation}







Known properties:

1. strongly connected component
2. 