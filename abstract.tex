\begin{abstract}



\begin{tabular}{l l}

\textbf{Title}: & Network modelling of credit concentration risk\\
\\
\textbf{Candidate Number:} & 187489\\
\\
\textbf{Submitted for:}& MSc in Mathematical Finance\\
& Michaelmas Term 2016 \\
\end{tabular}


\vspace{1cm}


Banking regulators (Basel II) require market participants to identify and monitor concentration risk in their credit portfolios. Concentration risk is the risk arising from high exposure to single risk factors. Standard concentration measures rely mainly on the aggregation of the measures of each portfolio element and, for example sectorial allocation~\cite{lutkebohmert2008concentration}.
They underestimate, however, direct economic links between the elements of the portfolio and therefore the underlying structure of the portfolio.
The importance of structural measures has been confirmed by recent financial events, where large losses arising from worldwide contagion effects were registered~\cite{Kazi:2013vr}.
	
Network models are especially apt at identifying structural properties~\cite{newman2010networks}, as they take not only model the elements of a system (the nodes or vertices) but also the relations between those elements (the edges). Network models are proving to be useful tools for providing early-warning signals of systemic risk~\cite{Squartini:2013ev}, measuring liquidity risk~\cite{Karas:2012tp}, identifying sectors from time-series correlations~\cite{Onnela:2004vz,Fenn:2009uf,Fenn:2011kp}, as well as insights into finding diversified baskets of assets in the classical investment framework~\cite{Pozzi:2013ci}.

Previous work on concentration risk with network model~\cite{Sindel:2009vd} used a ramping parameter for eliminating correlations (connections) that have a smaller strength than the ramping parameter, and measured the community structure of the resulting network. The evolution of the number of isolated components with the ramping parameter provides an insight into how strongly interdependent the portfolio is.

While this method is able to measure the properties of specific portfolios, the expected properties of the model for a given portfolios aren't known yet.
The main goal of this thesis this is to study the properties of the ramping parameter under the most common theoretical graph models: the random graph model and the configuration model.
Studying the properties of the ramping parameter under these models has proved for other types of questions (consider for example of epidemiology models~\cite{keeling2005networkspublisher}) useful for several reasons.
Firstly, having the model inspected theoretically allows for a better understanding of its properties.
Secondly, even though the theoretical models do not capture all the properties of real world graphs, it provides useful baselines for comparison with the results with real portfolios.


\end{abstract}
