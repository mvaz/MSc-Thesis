\begin{abstract}



\begin{tabular}{l l}

\textbf{Title}: & Network modelling of credit concentration risk\\
\\
\textbf{Candidate Number:} & 187489\\
\\
\textbf{Submitted for:}& MSc in Mathematical Finance\\
& Michaelmas Term 2016 \\
\end{tabular}


\vspace{1cm}


Concentration risk is the risk arising from high exposure to single risk factors. Standard concentration measures rely mainly on the aggregation of the measures of each portfolio element at, for example the sectorial level~\cite{lutkebohmert2008concentration}.
They underestimate, however, direct economic links between the elements of the portfolio and therefore the underlying structure of the portfolio.

% The importance of structural measures has been confirmed by recent financial events, where large losses arising from worldwide contagion effects were registered~\cite{Kazi:2013vr}.
% Network models are especially apt at identifying structural properties~\cite{newman2010networks}, as they take not only model the elements of a system (the nodes or vertices) but also the relations between those elements (the edges).

This thesis builds on previous work on concentration risk with a network model~\cite{Sindel:2009vd}, which used a varying parameter for progressively eliminating interdependencies between obligors, and analysed the exposure of the component with largest exposure to measure the portfolio's concentration risk.
While this method is able to measure the properties of specific portfolios, the expected theoretical properties of the model have not been studied yet.

This thesis studies the theoretically expected distribution of the largest component in the sparsified correlation matrix by making a link between the varying parameter and the distribution function of the interobligor correlations, and interpreting these on the light of the random graph model.
For this, it makes the assumptions that (1) the exposure is homogeneously distributed among the obligors and that (2) the correlations between obligors are statistically independent.
Under these assumptions, we find that the distribution of the expected largest component is strongly dominated by the so-called giant component and therefore strongly dependent on the assumed distribution of the correlations.


\end{abstract}
